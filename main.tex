% !TEX TS-program = xelatex
% !TEX encoding = UTF-8 Unicode

% 雖然這裡把浮水印跟DOI在LaTeX內直接加入的功能保留(預設為關閉),
% 但強烈建議大家還是使用Acobat或Foxit PDF Reader手動嫁入,
% 避免格式跑掉,然後被圖書館找碴。

\documentclass[
  degree        = master,               % degree = master | doctor
  language      = chinese,              % language = chinese | english
  watermark     = false,                % watermark = true | false
  doi           = false,                % doi = true | false
  AutoFakeSlant = 0.25,
  AutoFakeBold  = 2
]{ntuthesis}

% !TeX root = ./main.tex

% --------------------------------------------------
% 資訊設定(Information Configs)
% --------------------------------------------------

\ntusetup{
  university*   = {National Taiwan University},
  university    = {國立臺灣大學},
  college       = {社會科學院},
  college*      = {College of Social Sciences},
  institute     = {政治學系},
  institute*    = {Department of Poliotical Science},
  title         = {無詠唱魔法理論基礎},
  title*        = {Theoretical Foundations of Non-chanted Magic},
  author        = {魯迪烏斯·格雷拉特},
  author*       = {Rudeus Greyrat},
  ID            = {R09322008},
  advisor       = {洛琪希·米格路迪亞·格雷拉特},
  advisor*      = {Roxy Migurdia Greyrat},
  date          = {2023-08-01},         % 若註解掉,則預設為當天
  oral-date     = {2023-08-01},         % 若註解掉,則預設為當天
  DOI           = {10.5566/NTU2018XXXXX},
  keywords      = {LaTeX, 中文, 論文, 模板},
  keywords*     = {LaTeX, CJK, Thesis, Template},
}

% --------------------------------------------------
% 加載套件(Include Packages)
% --------------------------------------------------

\usepackage[style = apa, 
            backend = biber, 
            natbib]{biblatex}           % 參考文獻      
\usepackage{paralist}                   % 列表環境
\usepackage{lipsum}                     % 英文亂字
\usepackage{zhlipsum}                   % 中文亂字
\usepackage{url}
\usepackage{adjustbox}                  % 圖表縮放
\usepackage{threeparttable}             % 複雜圖表環境
\usepackage{rotating}                   % 表格水平轉置
\usepackage{tablefootnote}              % 表格內註解
\usepackage{xcolor}                     % 顏色RGB
\usepackage{tikz}                       % LaTeX畫圖
\usepackage{makecell}
\usepackage{enumitem}

% \usepackage[capposition = top]{floatrow}

% --------------------------------------------------
% 套件設定(Packages Settings)
% --------------------------------------------------

\addbibresource{back/references.bib}         % 參考文獻資源庫
\usetikzlibrary{shapes, arrows, positioning} % LaTeX畫圖資源庫

% 自訂數學符號
\DeclareMathOperator*{\argmin}{arg\,min}

% 自調表格種類
\newcolumntype{L}[1]{>{\raggedright\let\newline\\\arraybackslash\hspace{0pt}}m{#1}}
\newcolumntype{C}[1]{>{\centering\let\newline\\\arraybackslash\hspace{0pt}}m{#1}}
\newcolumntype{R}[1]{>{\raggedleft\let\newline\\\arraybackslash\hspace{0pt}}m{#1}}


\begin{document}

% 封面與口試審定
% Cover and Verification Letter
\frontmatter
\makecover                                     % 論文封面(Cover)
\makeverification{./front/verification.pdf}    % 口試委員審定書(Verification Letter)

% 致謝與論文摘要
% Acknowledgement and Abstract
% !TeX root = ../main.tex

\begin{acknowledgement}

我已經能夠走向外界。她成功達成沒有任何人能辦到的事情。這是生前,父母跟兄弟們都沒能辦到的事情。洛琪希卻幫我辦到了。不是靠不負責任的空口白話,而是扛起責任帶給我勇氣。她並非刻意這麼做。我很清楚。她是為了自己。這點我也知道。不過,我還是該尊敬她。要尊敬那位嬌小的少女。我在內心發誓,目送洛琪希的背影遠去直到完全消失。手裡只剩下她送我的魔杖跟項鏈。還有許許多多的知識。--- \textit{\Kai 《無職轉生》,第一卷}

\end{acknowledgement}           % 致謝(Acknowledgement)
\input{front/abstract}                  % 摘要(Abstract)

% 生成目錄與符號列表
% Contents of Tables and Denotation
\maketableofcontents                    % 目錄(Table of Contents)
\makelistoffigures                      % 圖目錄(List of Figures)
\makelistoftables                       % 表目錄(List of Tables)

% 論文內容
% Contents of Thesis
\mainmatter
% !TeX root = ../main.tex

\chapter{緒論}

\section{文件說明}

這個模板根據\href{https://github.com/Hsins}{Hsins}大的\href{https://github.com/Hsins/NTU-Thesis-LaTeX-Template}{NTU-Thesis-LaTeX-Template}所修改而來,並供臺大政治系研究生利用\href{https://www.latex-project.org}{\LaTeX}撰寫論文。這個版本的模板大致上仍為持Hsins大所設計架構,並只有在以下幾處進行格式上的調整來更符合政治系的論文格式。你可以在我的\href{https://github.com/withworksc/NTUPS_Thesis_Template}{Github專案}中下載原始碼\footnote{\texttt{href}字的顏色預設仍為黑色,如果有需要可以到cls裡面加回來。}。
% !TeX root = ../main.tex

\chapter{文獻回顧}

\section{英文引用測試}

\lipsum[2]\citep{hetherington_2001_partisan,layman_geoffrey_2002_conflict,layman_et_al_2006_polar,zaller_1992_mass,iyengar_et_al_2012_affective,iyengar_et_al_2019_affective}.

\section{中文引用測試}

以下為中文引用\citep{sheng_2010_longtitude,chen_et_al_2009_cross,hawang_2004_legi,sheng_2003_compare,juang_et_al_2017_curroption,}。
% !TeX root = ../main.tex

\chapter{研究設計與方法}

\section{模型設定}

M-Estimation的目標試如等式\eqref{eq:m_esti}所示:
\begin{equation}
\label{eq:m_esti}
\hat{\theta} = \argmin_{\theta \in \Theta} \sum_{i = 1}^n \nabla_{\theta} q (w_i, \theta) \triangleq \argmin_{\theta \in \Theta} \sum_{i = 1}^n s_i(\theta)
\end{equation}


% !TeX root = ../main.tex

\chapter{實證結果}

\section{插入圖表}


% Table created by stargazer v.5.2.2 by Marek Hlavac, Harvard University. E-mail: hlavac at fas.harvard.edu
% Date and time: Wed, May 12, 2021 - 23:43:54
\begin{table}[!htbp] \centering 
  \caption{迴歸報表範例} 
  \label{tb:regTable} 
\begin{tabular}{@{\extracolsep{5pt}}lcccc} 
\\[-1.8ex]\hline 
\hline \\[-1.8ex] 
 & \multicolumn{4}{c}{\textit{Dependent variable:}} \\ 
\cline{2-5} 
\\[-1.8ex] & \multicolumn{4}{c}{logRent} \\ 
 & POLS & RE & FE & FD \\ 
\\[-1.8ex] & (1) & (2) & (3) & (4)\\ 
\hline \\[-1.8ex] 
 y90 & 0.262$^{***}$ & 0.323$^{***}$ & 0.386$^{***}$ &  \\ 
  & (0.035) & (0.029) & (0.037) &  \\ 
  & & & & \\ 
 logPop & 0.041$^{*}$ & 0.056$^{*}$ & 0.072 & 0.072 \\ 
  & (0.023) & (0.029) & (0.088) & (0.088) \\ 
  & & & & \\ 
 logAvgInc & 0.571$^{***}$ & 0.447$^{***}$ & 0.310$^{***}$ & 0.310$^{***}$ \\ 
  & (0.053) & (0.052) & (0.066) & (0.066) \\ 
  & & & & \\ 
 pctstu & 0.005$^{***}$ & 0.005$^{***}$ & 0.011$^{***}$ & 0.011$^{***}$ \\ 
  & (0.001) & (0.001) & (0.004) & (0.004) \\ 
  & & & & \\ 
 Constant & $-$0.569 & 0.445 &  & 0.386$^{***}$ \\ 
  & (0.535) & (0.566) &  & (0.037) \\ 
  & & & & \\ 
\hline \\[-1.8ex] 
Observations & 128 & 128 & 128 & 64 \\ 
Adjusted R$^{2}$ & 0.857 & 0.949 & 0.950 & 0.288 \\ 
\hline 
\hline \\[-1.8ex] 
\textit{Note:}  & \multicolumn{4}{r}{$^{*}$p$<$0.1; $^{**}$p$<$0.05; $^{***}$p$<$0.01} \\ 
\end{tabular} 
\end{table} 


\begin{figure}[!htbp]
\centering
\caption{全台家戶薪資分布,按村里}
\label{fig:income}
\includegraphics[width = \textwidth]{contents/figures/township_income.png}
\end{figure}



% 參考文獻
% References
\refmatter
% \printbibliography                   % <- 你可以直接把中英的參考文獻混再一起

\printbibheading                       % <- 或者分中文與英文兩部分,但記得在reference.bib中要加入keyword
\section*{一、英文部分}
\printbibliography[keyword = {english}, heading = none]
\section*{二、中文部分}
\printbibliography[keyword = {chinese}, heading = none]

% 附錄
% Appendices
% !TeX root = ../main.tex

\appendix{A}{這是附錄}
\section{附錄段落1}
\section{附錄段落2}

% \input{back/appendix02}               % <- 我先把原有的附錄先註解到了,你如果需要可以再重新加回

\end{document}
